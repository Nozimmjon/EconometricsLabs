\documentclass[10pt,landscape]{article}
\usepackage{multicol}
\usepackage{calc}
\usepackage{ifthen}
\usepackage{amsmath}
\usepackage{lipsum}
\usepackage{booktabs}
\usepackage{xcolor}
\usepackage{threeparttable}
\usepackage[landscape]{geometry}

\definecolor{oucrimson}{RGB}{132,22,23} % OU crimson
% \definecolor{navy}{RGB}{132,22,23} % OU crimson


\usepackage{hyperref}
\hypersetup{
    unicode=true,
    bookmarksnumbered=true,
    bookmarksopen=true,
    bookmarksopenlevel=3,
    breaklinks=true, 
    pdfborder={0 0 0},
    colorlinks,
    citecolor=oucrimson,
    filecolor=oucrimson,
    linkcolor=oucrimson,
    urlcolor=oucrimson
}

% \makeatletter
% \renewcommand\verbatim@font{\itshape}
% \makeatother

% \makeatletter
% \renewcommand\verbatim@font{\itshape}
% \makeatother


% To make this come out properly in landscape mode, do one of the following
% 1.
%  pdflatex latexsheet.tex
%
% 2.
%  latex latexsheet.tex
%  dvips -P pdf  -t landscape latexsheet.dvi
%  ps2pdf latexsheet.ps


% If you're reading this, be prepared for confusion.  Making this was
% a learning experience for me, and it shows.  Much of the placement
% was hacked in; if you make it better, let me know...


% This sets page margins to .5 inch if using letter paper, and to 1cm
% if using A4 paper. (This probably isn't strictly necessary.)
% If using another size paper, use default 1cm margins.
\ifthenelse{\lengthtest { \paperwidth = 11in}}
	{ \geometry{top=.5in,left=.5in,right=.5in,bottom=.5in} }
	{\ifthenelse{ \lengthtest{ \paperwidth = 297mm}}
		{\geometry{top=1cm,left=1cm,right=1cm,bottom=1cm} }
		{\geometry{top=1cm,left=1cm,right=1cm,bottom=1cm} }
	}

% Turn off header and footer
\pagestyle{empty}
 

% Redefine section commands to use less space
\makeatletter
\renewcommand{\section}{\@startsection{section}{1}{0mm}%
                                {-1ex plus -.5ex minus -.2ex}%
                                {0.5ex plus .2ex}%x
                                {\normalfont\large\bfseries}}
\renewcommand{\subsection}{\@startsection{subsection}{2}{0mm}%
                                {-1explus -.5ex minus -.2ex}%
                                {0.5ex plus .2ex}%
                                {\normalfont\normalsize\bfseries}}
\renewcommand{\subsubsection}{\@startsection{subsubsection}{3}{0mm}%
                                {-1ex plus -.5ex minus -.2ex}%
                                {1ex plus .2ex}%
                                {\normalfont\small\bfseries}}
\makeatother

% Define BibTeX command
\def\BibTeX{{\rm B\kern-.05em{\sc i\kern-.025em b}\kern-.08em
    T\kern-.1667em\lower.7ex\hbox{E}\kern-.125emX}}

% Don't print section numbers
\setcounter{secnumdepth}{0}


\setlength{\parindent}{0pt}
\setlength{\parskip}{0pt plus 0.5ex}


% -----------------------------------------------------------------------

\begin{document}

\raggedright
\footnotesize
\begin{multicols}{3}


% multicol parameters
% These lengths are set only within the two main columns
%\setlength{\columnseprule}{0.25pt}
\setlength{\premulticols}{1pt}
\setlength{\postmulticols}{1pt}
\setlength{\multicolsep}{1pt}
\setlength{\columnsep}{2pt}

\begin{center}
     {\Large{\textbf{Econometrics in R's \texttt{tidyverse}}}} \\
     
     {\footnotesize by Tyler Ransom, University of Oklahoma}
     
     {\footnotesize \href{https://www.twitter.com/tyleransom}{\makeatletter @tyleransom \makeatother}}
\end{center}

\section{Basics of R}
R can be thought of as a really fancy calculator

\smallskip{}

\textbf{Packages:}\\
\begin{itemize}
    \item R comes with a lot of functionality out-of-the-box
    \item Other functionality requires the user to load packages
    \item One-time installation: \verb!install.packages("tidyverse")!
    \item Each time you open R: \verb!library(tidyverse)!
\end{itemize}

\smallskip{}

\textbf{Commenting:}\\
\begin{itemize}
    \item Use \verb!#! to make a comment
    \item This tells R to ignore that code
    \item[] \verb!# My name is Tyler!
\end{itemize}

\smallskip{}

\textbf{Assignment operator:}\\
\begin{itemize}
    \item Use \verb!<-! to store a calculation, e.g. \verb!x <- 3! (``$x=3$'')
\end{itemize}

\smallskip{}

\textbf{Pipe operator:}\\
\begin{itemize}
    \item Use \verb!%>%! to ``pipe'' objects
    \item[] \verb!y <- mean(log(x))! becomes \verb!y <- x %>% log %>% mean!
    \item \verb!%<>%! pipes forward, then backwards
    \item[] \verb!x <- mean(log(x))! is same as \verb!x %<>% log %>% mean!
\end{itemize}

\section{Working with Data}
R's fundamental data object is a \textbf{data frame}

\smallskip{}

Like spreadsheets, stores data in columns and rows

\smallskip{}

\verb!tidyverse! uses \textbf{tibbles} (enhanced data frames)
\begin{itemize}
    \item[] \verb!df <- as_tibble(mtcars)!
\end{itemize}

\smallskip{}

\textbf{Reading in data}\\
\begin{itemize}
    \item Many functions for reading in different types of data
    \item[] \verb!df <- read_csv("myfile.csv")! (comma separated)
    \item[] \verb!df <- read_fwf("myfile.dat")! (fixed-width)
    \item More details: see \href{https://rawgit.com/rstudio/cheatsheets/master/data-import.pdf}{Data Importing Cheat Sheet}
    \item \verb!haven! package: import foreign files (e.g. SAS, Stata, ...)
\end{itemize}

\smallskip{}

\textbf{Accessing columns of data}\\
\begin{itemize}
    \item To reference a column in a tibble, use \verb!$!
    \item[] \verb!df$mpg!
    \item[] \verb!mean(df$mpg)! will return sample avg of \verb!mpg! variable
\end{itemize}

\smallskip{}

\textbf{Ignore missing values}\\
\begin{itemize}
    \item Missing values are indicated by \verb!NA!
    \item Some commands won't automatically ignore \verb!NA! values
    \item For these cases, use \verb!na.rm! option
    \item[] \verb!mean(df$mpg, na.rm=TRUE)!
    \item[] \verb!df$mpg %>% mean(na.rm=TRUE)! (equivalent)
    \item Otherwise, \texttt{R} would say the mean is \verb!NA!
\end{itemize}

\smallskip{}

\textbf{Removing columns and rows from a tibble}\\
\begin{itemize}
    \item To keep columns in a tibble, use \verb!select()!
    \item[] \verb!df1 <- df %>% select(mpg,disp,hp,gear,carb)!
    \item To keep rows in a tibble, use \verb!filter()!
    \item[] \verb!df1 %<>% filter(mpg>=10)!
    \item To remove columns, put a minus in front
    \item[] \verb!df1 <- df %>% select(-mpg,-disp)!
\end{itemize}

\smallskip{}

\textbf{Remove missing values from a tibble}\\
\begin{itemize}
    \item To remove \textbf{all} rows with \textit{any} \verb!NA! values, use \verb!drop_na()!
    \item[] \verb!df1 <- df %>% drop_na()!
    \item Can also drop \verb!NA!'s from particular columns:
    \item[] \verb!df1 <- df %>% drop_na(gear,carb)!
\end{itemize}

\smallskip{}

\textbf{Creating new columns in a tibble}\\
\begin{itemize}
    \item To create a new column in a tibble, use \verb!mutate()!
    \item[] \verb!df1 %<>% mutate(mpg.squared = mpg^2)!
\end{itemize}

\smallskip{}

\textbf{Manipulating values of a variable}\\
\begin{itemize}
    \item To replace (i.e. recode) values of a variable:
    \item[] \verb!df %<>% mutate(gear = replace(gear,gear==4,99))!
    \item[] Changes all 4's in \verb!gear! to be 99's
    \item[] \verb!gear==4! can be any other logical condition
    \item To specify a series of conditions, use \verb!%in%!
    \item[] \verb!df %<>% mutate(hp =!
    \item[] \verb!replace(hp,hp %in% c(110,120),99))!
    \item[] Changes all 110's or 120's in \verb!hp! to be 99's
\end{itemize}

\smallskip{}

\textbf{Working with discrete variables}\\
\begin{itemize}
    \item Discrete variables often require special treatment
    \item In R, declare discrete variables as \textbf{factors}
    \item[] \verb!df %<>% mutate(gear = as.factor(gear))!
\end{itemize}

\smallskip{}

\textbf{Other data manipulations}\\
\begin{itemize}
    \item See \href{https://www.rstudio.com/wp-content/uploads/2015/02/data-wrangling-cheatsheet.pdf}{Data Wrangling Cheat Sheet}
\end{itemize}

\section{Getting to know your data}
It's important to know what's in your data by
\begin{enumerate}
    \item Looking at summary statistics
    \item Performing cross-tabulations
    \item Visualizing certain variables
\end{enumerate}

\smallskip{}

\textbf{Summary statistics}\\
\begin{itemize}
    \item Report quartiles, min/max, mean, and \#\verb!NA!'s:
    \item[] \verb!summary(df)!
    \item Report N (total non-missing), mean, SD, min/max:
    \item[] \verb!df %>% as.data.frame %>% stargazer(type="text")!
\end{itemize}

\smallskip{}

\textbf{Cross-tabulations}\\
\begin{itemize}
    \item Report frequencies of a discrete variable:
    \item[] \verb!table(df$gear)!
    \item Average $y$ by categories of a discrete $x$ variable:
    \item[] \verb!df %>% group_by(gear) %>% !
    \item[] \verb!summarize(m.mpg = mean(mpg))!
\end{itemize}

\smallskip{}

\textbf{Visualization}\\
\begin{itemize}
    \item Often helpful to look at a histogram or line graph
    \item Histogram (continuous $x$):
    \item[] \verb!ggplot(df, aes(mpg)) + geom_histogram()!
    \item Histogram (factor $x$):
    \item[] \verb!ggplot(df,aes(x=gear)) + geom_bar()!
    \item Kernel density plot:
    \item[] \verb!ggplot(df, aes(mpg)) + geom_density()!
    \item Simple scatter plot with linear fit:
    \item[] \verb!ggplot(df,aes(disp,mpg)) + geom_point() +!
    \item[] \verb!geom_smooth(method="lm")!
    \item More details: see \href{https://www.rstudio.com/wp-content/uploads/2015/03/ggplot2-cheatsheet.pdf}{ggplot2 Cheat Sheet}
\end{itemize}


\section{Regression modeling}

\smallskip{}

\textbf{Basic OLS regression}\\
\begin{itemize}
    \item Regression:
    \item[] \verb!est <- lm(mpg ~ gear + hp, data=df)!
    \item Examine regression output:
    \item[] \verb!summary(est)!
    \item[] \verb!tidy(est)!
    \item[] \verb!stargazer(est,type="text")!
    \item Other functional forms:
    \item[] \verb!est <- lm(mpg ~ gear + I(gear^2), data=df)!
    \item[] \verb!est <- lm(log(mpg) ~ gear + I(gear^2), data=df)!
    \item Factor variables automatically get separate intercepts
\end{itemize}

\smallskip{}

\textbf{$t$-statics and $F$-statistics}\\
\begin{itemize}
    \item $t$-stats, $p$-values reported in regression output
    \item $F$-test:
    \item[] \verb!linearHypothesis(est,c("gear","hp"))!
    \item[] tests $H_0: \beta_{gear} = 0, \beta_{hp}=0$
    \item[] \verb!linearHypothesis(est,c("gear=5","hp=-1"))!
    \item[] tests $H_0: \beta_{gear} = 5, \beta_{hp}=-1$
    \item Robust $F$-test (see next section):
    \item[] \verb!linearHypothesis(est.rob,c("gear","hp"))!
\end{itemize}

\smallskip{}

\textbf{Robust standard errors (\texttt{estimatr} package)}\\
\begin{itemize}
    \item Correct for heteroskedasticity:
    \item[] \verb!est.rob <- lm_robust(mpg ~ gear + hp, data=df)!
    
    or
    \item[] \verb!stargazer(est,se=starprep(est.rob),type="text")!
    \item Correct for serial correlation:
    \item[] \verb!fixed.est <- est %>% coeftest(vcov=NeweyWest)!
    \item[] \verb!stargazer(est,se=list(fixed.est[,2]),type="text")!
    \item Correct for clustering:
    \item[] \verb!est.clust <- lm_robust(mpg ~ gear + hp, data=df,!
    \item[] \verb!clusters=df$carb)!
    
    or
    \item[] \verb!stargazer(est,se=starprep(est.clust),type="text")!
\end{itemize}




\section{Instrumental Variables}

\smallskip{}

\begin{itemize}
    \item Let \verb!drat! be the endogenous covariate
    \item Let \verb!wt! be the instrument
    \item Let \verb!qsec! and \verb!gear! be exogenous covariates
    \item[] \verb!est.iv <- ivreg(mpg ~ drat + qsec + gear | !
    \item[] \verb!wt + qsec + gear, data=df)!
    \item Instruments come after the \verb!|! symbol
    \item Endogenous covariates come before the \verb!|! symbol
    \item Exogenous covariates appear on both sides of the \verb!|!
    \item First-stage regression:
    \item[] \verb!est.1 <- lm(drat ~ wt + qsec + gear, data=df)!
    \item[] \verb!df %<>% mutate(drat.hat = est.1$fitted.values)!
    \item Second-stage regression:
    \item[] \verb!est.2 <- lm(mpg ~ drat.hat + qsec+ gear,data=df)!
    \item Can also use \verb!estimatr! for robust SEs:
    \item[] \verb!est.ivr <- iv_robust(mpg ~ drat + qsec + gear | !
    \item[] \verb!wt + qsec + gear, data=df)!
\end{itemize}




\section{Working with time series data}

\smallskip{}

\begin{itemize}
    \item Declare a time series data frame
    \item[] \verb!df.ts <- zoo(df, order.by=df$year)!
    \item Time series line plot:
    \item[] \verb!ggplot(df.ts, aes(year, inf)) + geom_line()!
    \item Simple AR(1) model:
    \item[] \verb!est <- dynlm(inf ~ L(inf,1), data=df.ts)!
    \item First-differences model:
    \item[] \verb!est.diff <- dynlm(d(inf) ~ unem, data = df.ts)!
    \item ADF test for unit root:
    \item[] \verb!adf.test(df1.ts$inf, k=1)!
    \item ARIMA model:
    \item[] \verb!est.arima <- auto.arima(df.ts$inf)!
    \item Plot $h$-period-ahead forecast intervals
    \item[] \verb!autoplot(forecast(est.arima, h=2))!
    \item Extended date and time functions available in \href{https://evoldyn.gitlab.io/evomics-2018/ref-sheets/R_lubridate.pdf}{\texttt{lubridate} package}
\end{itemize}




\section{Working with panel data}

\smallskip{}

\begin{itemize}
    \item Report number of units and time periods
    \item[] \verb!pdim(df)!
    \item Pooled OLS model
    \item[] \verb!est.pols <- plm(lwage ~ exper + I(exper^2) +! 
    \item[] \verb!year, data = df, index = c("id","year"),! 
    \item[] \verb!model = "pooling")!
    \item Random effects model
    \item[] \verb!est.re <- plm(lwage ~ exper + I(exper^2) +! 
    \item[] \verb!year, data = df, index = c("id","year"),! 
    \item[] \verb!model = "random")!
    \item Fixed effects model
    \item[] \verb!est.fe <- plm(lwage ~ exper + I(exper^2) +! 
    \item[] \verb!year, data = df, index = c("id","year"),! 
    \item[] \verb!model = "within")!
    \item First differences model
    \item[] \verb!est.fd <- plm(lwage ~ exper + I(exper^2) +! 
    \item[] \verb!year, data = df, index = c("id","year"),! 
    \item[] \verb!model = "fd")!
\end{itemize}




\section{Limited dependent variable models}

\smallskip{}

\textbf{Linear probability model (LPM):}\\
\begin{itemize}
    \item If $y$ is a factor, format it as a numeric
    \item[] \verb!est.lpm <- lm(as.numeric(y) ~ x1 + x2, data=df)!
\end{itemize}

\textbf{Logit and Probit:}\\

\smallskip{}

In this case, $y$ should be formatted as a factor

\smallskip{}

\begin{itemize}
    \item Logit:
    \item[] \verb!est.logit <- glm(y ~ x1 + x2,!
    \item[] \verb!family=binomial(link="logit"),data=df)!
    \item Probit:
    \item[] \verb!est.probit <- glm(y ~ x1 + x2,!
    \item[] \verb!family=binomial(link="probit"),data=df)!
\end{itemize}


% \textcolor{white}{\lipsum[1-12]}



\section{List of packages}
The document requires the following packages:

\smallskip{}

\begin{multicols}{4}
\verb!tidyverse!  \\
\verb!magrittr!  \\
\verb!stargazer!  \\
\verb!broom!  \\
\verb!car!  \\
\verb!estimatr!  \\
\verb!lmtest!  \\
\verb!clubSandwich!  \\
\verb!sandwich!  \\
\verb!zoo!  \\
\verb!dynlm!  \\
\verb!AER!  \\
\verb!tseries!  \\
\verb!lubridate!  \\
\verb!forecast!  \\
\verb!plm!  \\
\end{multicols}

\rule{0.3\linewidth}{0.25pt}
\scriptsize

Layout by Winston Chang, \href{http://wch.github.io/latexsheet/}{http://wch.github.io/latexsheet/}


\end{multicols}
\end{document}
