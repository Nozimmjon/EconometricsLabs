\documentclass[10pt,landscape]{article}
\usepackage{multicol}
\usepackage{calc}
\usepackage{ifthen}
\usepackage{amsmath}
\usepackage{lipsum}
\usepackage{booktabs}
\usepackage{xcolor}
\usepackage{threeparttable}
\usepackage[landscape]{geometry}

\definecolor{oucrimson}{RGB}{132,22,23} % OU crimson

\usepackage{hyperref}
\hypersetup{
    unicode=true,
    bookmarksnumbered=true,
    bookmarksopen=true,
    bookmarksopenlevel=3,
    breaklinks=true, 
    pdfborder={0 0 0},
    colorlinks,
    citecolor=oucrimson,
    filecolor=oucrimson,
    linkcolor=oucrimson,
    urlcolor=oucrimson
}


% To make this come out properly in landscape mode, do one of the following
% 1.
%  pdflatex latexsheet.tex
%
% 2.
%  latex latexsheet.tex
%  dvips -P pdf  -t landscape latexsheet.dvi
%  ps2pdf latexsheet.ps


% If you're reading this, be prepared for confusion.  Making this was
% a learning experience for me, and it shows.  Much of the placement
% was hacked in; if you make it better, let me know...


% This sets page margins to .5 inch if using letter paper, and to 1cm
% if using A4 paper. (This probably isn't strictly necessary.)
% If using another size paper, use default 1cm margins.
\ifthenelse{\lengthtest { \paperwidth = 11in}}
	{ \geometry{top=.5in,left=.5in,right=.5in,bottom=.5in} }
	{\ifthenelse{ \lengthtest{ \paperwidth = 297mm}}
		{\geometry{top=1cm,left=1cm,right=1cm,bottom=1cm} }
		{\geometry{top=1cm,left=1cm,right=1cm,bottom=1cm} }
	}

% Turn off header and footer
\pagestyle{empty}
 

% Redefine section commands to use less space
\makeatletter
\renewcommand{\section}{\@startsection{section}{1}{0mm}%
                                {-1ex plus -.5ex minus -.2ex}%
                                {0.5ex plus .2ex}%x
                                {\normalfont\large\bfseries}}
\renewcommand{\subsection}{\@startsection{subsection}{2}{0mm}%
                                {-1explus -.5ex minus -.2ex}%
                                {0.5ex plus .2ex}%
                                {\normalfont\normalsize\bfseries}}
\renewcommand{\subsubsection}{\@startsection{subsubsection}{3}{0mm}%
                                {-1ex plus -.5ex minus -.2ex}%
                                {1ex plus .2ex}%
                                {\normalfont\small\bfseries}}
\makeatother

% Define BibTeX command
\def\BibTeX{{\rm B\kern-.05em{\sc i\kern-.025em b}\kern-.08em
    T\kern-.1667em\lower.7ex\hbox{E}\kern-.125emX}}

% Don't print section numbers
\setcounter{secnumdepth}{0}


\setlength{\parindent}{0pt}
\setlength{\parskip}{0pt plus 0.5ex}


% -----------------------------------------------------------------------

\begin{document}

\raggedright
\footnotesize
\begin{multicols}{3}


% multicol parameters
% These lengths are set only within the two main columns
%\setlength{\columnseprule}{0.25pt}
\setlength{\premulticols}{1pt}
\setlength{\postmulticols}{1pt}
\setlength{\multicolsep}{1pt}
\setlength{\columnsep}{2pt}

\begin{center}
     {\Large{\textbf{Econometrics Cheat Sheet}}} \\
     
     {\footnotesize by Tyler Ransom, University of Oklahoma}
     
     {\footnotesize \href{https://www.twitter.com/tyleransom}{\makeatletter @tyleransom \makeatother}}
\end{center}

\section{Data \& Causality}
Basics about data types and causality.

\subsection{Types of data}
\newlength{\MyLen}
\settowidth{\MyLen}{\texttt{letterpaper}/\texttt{a4paper} \ }
\begin{tabular}{@{}p{\the\MyLen}%
                @{}p{\linewidth-\the\MyLen}@{}}
Experimental            & Data from randomized experiment \\
Observational           & Data collected passively \\
Cross-sectional         & Multiple units, one point in time \\
Time series             & Single unit, multiple points in time \\
Longitudinal (or Panel) & Multiple units followed over multiple time periods \\
\end{tabular}

\textbf{Experimental data} 
\begin{itemize}
\item Correlation $\implies$ Causality
\item Very rare in Social Sciences
\end{itemize}






\section{Statistics basics}
We examine a \textbf{random sample} of data to learn about the population

\medskip{}

\begin{tabular}{@{}p{\the\MyLen}%
                @{}p{\linewidth-\the\MyLen}@{}}
Random sample         &  Representative of population \\
Parameter $(\theta)$  &  Some number describing population  \\
Estimator of $\theta$ &  Rule assigning value of $\theta$ to sample    \\
                      &  e.g. Sample average, $\overline{Y} = \frac{1}{N}\sum_{i=1}^{N} Y_{i}$ \\
Estimate of $\theta$  &  What the estimator spits out \par
                         for a particular sample $(\hat{\theta})$   \\
Sampling distribution & Distribution of estimates \par
                        across all possible samples \\
Bias of estimator $W$ & $E\left(W\right)-\theta$ \\
Efficiency            & $W$ efficient if $Var(W)<Var(\widetilde{W})$ \\
Consistency           & $W$ consistent if $\hat{\theta}\rightarrow\theta$ as $N\rightarrow\infty$ \\
\end{tabular}

\subsection{Hypothesis testing}
The way we answer yes/no questions about our population using a sample of data. e.g. ``Does increasing public school spending increase student achievement?''

\medskip{}

\begin{tabular}{@{}p{\the\MyLen}%
                @{}p{\linewidth-\the\MyLen}@{}}
null hypothesis $(H_0)$       & Typically, $H_0: \theta=0$ \\
alt. hypothesis $(H_a)$       & Typically, $H_0: \theta\neq 0$ \\
significance level $(\alpha)$ & Tolerance for making Type I error; \par (e.g. 10\%, 5\%, or 1\%)\\
test statistic $(T)$          & Some function of the sample of data \\
critical value $(c)$          & Value of $T$ such that reject $H_0$ if $\vert T \vert > c$; \par $c$ depends on $\alpha$; \par $c$ depends on if 1- or 2-sided test\\
$p$-value                     & Largest $\alpha$ at which fail to reject $H_0$; \par reject $H_0$ if $p<\alpha$ \\
\end{tabular}

% \medskip{}

% \begin{align*}
% t = & \frac{\text{estimate} - \text{null}}{\text{std. err.}} \\
% \end{align*}

\section{Simple Regression Model}
Regression is useful because we can estimate a \textit{ceteris paribus} relationship between some variable $x$ and our outcome $y$

\begin{align*}
y=\beta_{0}+\beta_{1}x+u
\end{align*}

We want to estimate $\hat{\beta}_1$, which gives us the effect of $x$ on $y$.

\subsection{OLS formulas}
To estimate $\hat{\beta}_0$ and $\hat{\beta}_1$, we make two assumptions:
\begin{enumerate}
\item $E\left(u\right) = 0$
\item $E\left(u\vert x\right) = E\left(u\right)$ for all $x$
\end{enumerate}

When these hold, we get the following formulas:
\begin{align*}
\hat{\beta}_{0}&=\overline{y}-\hat{\beta}_{1}\overline{x} \\
\hat{\beta}_{1}&=\frac{\widehat{Cov}\left(y,x\right)}{\widehat{Var}\left(x\right)}
\end{align*}

\begin{tabular}{@{}p{\the\MyLen}%
                @{}p{\linewidth-\the\MyLen}@{}}
fitted values $(\hat{y}_i)$     & $\hat{y}_{i} = \hat{\beta}_0 + \hat{\beta}_1 x_{i}$ \\
residuals $(\hat{u}_i)$         & $\hat{u}_{i} = y_{i} - \hat{y}_i$ \\
Total Sum of Squares            & $SST=\sum_{i=1}^{N}(y_{i}-\overline{y})^{2}$ \\
Expl. Sum of Squares            & $SSE =\sum_{i=1}^{N}(\hat{y}_{i}-\overline{y})^{2}$\\
Resid. Sum of Squares           & $SSR =\sum_{i=1}^{N}\hat{u}_{i}^{2}$\\
$R$-squared ($R^2$)             & $R^{2}=\frac{SSE}{SST}$; \par ``frac. of var. in $y$ explained by $x$''\\
\end{tabular}

\subsection{Algebraic properties of OLS estimates}
\begin{enumerate}
    \item[] $\sum_{i=1}^{N}\hat{u}_{i}=0$ (mean \& sum of residuals is zero)
    \item[] $\sum_{i=1}^{N}x_{i}\hat{u}_{i}=0$ (zero covariance bet. $x$ and resids.)
    \item[] The OLS line (SRF) always passes through $\left(\overline{x},\overline{y}\right)$
    \item[] $SSE + SSR = SST$
    \item[] $0 \leq R^2 \leq 1$
\end{enumerate}

\subsection{Interpretation and functional form}
\begin{itemize}
    \item[] Our model is restricted to be \textbf{linear in parameters}
    \item[] But not linear in $x$
    \item[] Other functional forms can give more realistic model
\end{itemize}

\begin{center}
\begin{threeparttable}
\begin{tabular}{@{}l@{}c@{}c@{}l@{}}
    \toprule
    Model       & DV        & RHS        & Interpretation of $\beta_1$ \\ \midrule
    Level-level & $y$       & $x$         & $\phantom{\%}\Delta y = \beta_1 \Delta x$ \\
    Level-log   & $y$       & $\phantom{\%}\log(x)\phantom{\%}$   & $\phantom{\%}\Delta y = \left(\beta_1 / 100\right)\left[1\%\Delta x\right]\phantom{\%}$ \\
    Log-level   & $\phantom{\%}\log(y)\phantom{\%}$ & $x$         & $\%\Delta y = \left(100\beta_1\right)\Delta x$ \\
    Log-log     & $\log(y)$ & $\log(x)$   & $\%\Delta y = \beta_1 \%\Delta x$ \\ 
    Quadratic   & $y$       & $x+x^2$     & $\phantom{\%}\Delta y = \left(\beta_1+2\beta_2 x\right)\Delta x$ \\ \bottomrule
\end{tabular}
\scriptsize Note: $\text{DV }=\text{ dependent variable}$; $\text{RHS }=\text{ right hand side}$
\end{threeparttable}
\end{center}

\section{Multiple Regression Model}
Multiple regression is more useful than simple regression because we can more plausibly estimate \textit{ceteris paribus} relationships (i.e. $E\left(u\vert x\right) = E\left(u\right)$ is more plausible)
\begin{align*}
y=\beta_{0}+\beta_{1}x_1+\cdots+\beta_{k}x_k+u
\end{align*}

$\hat{\beta}_1,\ldots,\hat{\beta}_k$: \textbf{partial effect} of each of the $x$'s on $y$

\begin{align*}
\hat{\beta}_{0}&=\overline{y}-\hat{\beta}_{1}\overline{x}_1-\cdots-\hat{\beta}_{k}\overline{x}_k \\
\hat{\beta}_{j}&=\frac{\widehat{Cov}\left(y,\text{residualized }x_{j}\right)}{\widehat{Var}\left(\text{residualized }x_{j}\right)}
\end{align*}

where ``residualized $x_j$'' means the residuals from OLS regression of $x_j$ on all other $x$'s (i.e. $x_1, \ldots, x_{j-1}, x_{j+1}, \ldots x_k$)
    
\subsection{Gauss-Markov Assumptions}
\begin{enumerate}
    \item $y$ is a linear function of the $\beta$'s
    \item $y$ and $x$'s are randomly sampled from population
    \item No perfect multicollinearity
    \item $E\left(u\vert x_1, \ldots, x_k\right) = E\left(u\right) = 0$ (Unconfoundedness)
    \item $Var\left(u\vert x_1, \ldots, x_k\right) = Var\left(u\right) = \sigma^2$ (Homoskedasticity)
\end{enumerate}

\begin{tabular}{@{}p{\the\MyLen}%
                @{}p{\linewidth-\the\MyLen}@{}}
When (1)-(4) hold: & OLS is unbiased; i.e. $E(\hat{\beta}_j) = \beta_j$ \\
When (1)-(5) hold: & OLS is Best Linear Unbiased Estimator \\
\end{tabular}

\subsection{Variance of $u$ (a.k.a. ``error variance'')}
\begin{align*}
\hat{\sigma}^2 &= \frac{SSR}{N-K-1} \\
               &= \frac{1}{N-K-1}\sum_{i=1}^N \hat{u}_i^2
\end{align*}

\subsection{Variance and Standard Error of $\hat{\beta}_j$}
\begin{equation*}
Var(\hat{\beta}_{j})=\frac{\sigma ^{2}}{SST_{j}(1-R_{j}^{2})}\text{, }%
j=1,2,...,k
\end{equation*}
where
\begin{align*}
SST_{j}&=(N-1)Var(x_j) = \sum_{i=1}^N (x_{ij} - \overline{x}_j)\\
R_{j}^{2} &= R^2 \text{ from a regression of } x_j \text{ on all other } x \text{'s}
\end{align*}
\begin{tabular}{@{}p{\the\MyLen}%
                @{}p{\linewidth-\the\MyLen}@{}}
Standard deviation: & $\sqrt{Var}$ \\
Standard error:     & $\sqrt{\widehat{Var}}$ \\
\end{tabular}

\begin{equation*}
se(\hat{\beta}_{j})=\sqrt{\frac{\hat{\sigma} ^{2}}{SST_{j}(1-R_{j}^{2})}}, j=1,\ldots,k
\end{equation*}

\subsection{Classical Linear Model (CLM)}
Add a 6th assumption to Gauss-Markov:
\begin{enumerate}
    \setcounter{enumi}{5}
    \item $u$ is distributed $N\left(0,\sigma^2\right)$
\end{enumerate}

Need this to know what the \textit{distribution} of $\hat{\beta}_j$ is

Otherwise, can't conduct hypothesis tests about the $\beta$'s

\section{Testing Hypotheses about the $\beta$'s}
Under A (1)-(6), can test hypotheses about the $\beta$'s

\subsection{$t$-test for simple hypotheses}
To test a simple hypothesis like
\begin{align*}
H_0&:\beta_j = 0\\
H_a&: \beta_j \neq 0
\end{align*}
use a $t$-test:
\begin{align*}
t&=\frac{\hat{\beta}_j-0}{se\left(\hat{\beta}_j\right)}
\end{align*}
where $0$ is the null hypothesized value.

\medskip{}

Reject $H_0$ if $p<\alpha$ or if $\vert t \vert > c$ (See: Hypothesis testing)

\subsection{$F$-test for joint hypotheses}
Can't use a $t$-test for joint hypotheses, e.g.:
\begin{align*}
H_{0}&:\beta _{3}=0\text{, }\beta _{4}=0\text{, }\beta _{5}=0\\
H_{a}&:\beta _{3}\neq0\text{ OR }\beta _{4}\neq0\text{ OR }\beta _{5}\neq0
\end{align*}
Instead, use $F$ statistic:
\begin{align*}
F=\frac{(SSR_{r}-SSR_{ur})/(df_{r}-df_{ur})}{SSR_{ur}/df_{ur}}=\frac{(SSR_{r}-SSR_{ur})/q}{SSR_{ur}/(N-k-1)}
\end{align*}
where
\begin{align*}
SSR_r    &= SSR \text{ of restricted model (if }H_0\text{ true)} \\
SSR_{ur} &= SSR \text{ of unrestricted model (if }H_0\text{ false)} \\
q        &= \text{\# of equalities in }H_0 \\
N-k-1    &= \text{Deg. Freedom of unrestricted model} \\
\end{align*}
Reject $H_0$ if $p<\alpha$ or if $F > c$ (See: Hypothesis testing)

\medskip{}

Note: $F>0$, always

\section{Qualitative data}
\begin{itemize}
    \item Can use qualitative data in our model
    \item Must create a \textbf{dummy variable}
    \item e.g. ``Yes'' represented by 1 and ``No'' by 0
\end{itemize}

\textbf{dummy variable trap:} Perfect collinearity that happens when too many dummy variables are included in the model
\begin{align*}
y &= \beta_0 + \beta_1 happy + \beta_2 not\_happy + u
\end{align*}

The above equation suffers from the dummy variable trap because units can only be ``happy'' or ``not happy,'' so including both would result in perfect collinearity with the intercept

\subsection{Interpretation of dummy variables}
Interpretation of dummy variable coefficients is always relative to the excluded category (e.g. $not\_happy$):
\begin{align*}
y &= \beta_0 + \beta_1 happy + \beta_2 age + u
\end{align*}

\begin{itemize}
    \item[$\beta_1$:] avg. $y$ for those who are happy \textit{compared to} those who are unhappy, holding fixed age 
\end{itemize}

\subsection{Interaction terms}
\textbf{interaction term:} When two $x$'s are multiplied together
\begin{align*}
y &= \beta_0 + \beta_1 happy + \beta_2 age + \beta_3 happy \times age + u
\end{align*}

\begin{itemize}
    \item[$\beta_3$:] difference in $age$ \textbf{slope} for those who are happy \textit{compared to} those who are unhappy
\end{itemize}

\subsection{Linear Probability Model (LPM)}
When $y$ is a dummy variable, e.g.
\begin{align*}
happy &= \beta_0 + \beta_1 age + \beta_2 income + u
\end{align*}

$\beta$'s are interpreted as \textit{change in probability}:
\begin{align*}
\Delta\Pr(y=1) = \beta_1\Delta x
\end{align*}
By definition, homoskedasticity is violated in the LPM




\section{Time Series (TS) data}
\begin{itemize}
    \item Observe one unit over many time periods
    \item e.g. US quarterly GDP, 3-month T-bill rate, etc.
    \item New G-M assumption: no serial correlation in $u_t$
    \item Remove random sampling assumption (makes no sense)
\end{itemize}

\subsection{Two focuses of TS data}
\begin{enumerate}
    \item Causality (e.g. $\uparrow$ taxes $\overset{?}{\implies}$ $\downarrow$ GDP growth)
    \item Forecasting (e.g. AAPL stock price next quarter?)
\end{enumerate}

\subsection{Requirements for TS data}
To properly use TS data for causal inf / forecasting,\\need data free of the following elements:

\medskip{}

\begin{tabular}{@{}p{\the\MyLen}%
                @{}p{\linewidth-\the\MyLen}@{}}
Trends: & $y$ always $\uparrow$ or $\downarrow$ every period \\
Seasonality: & $y$ always $\uparrow$ or $\downarrow$ at regular intervals \\
Non-stationarity:  & $y$ has a unit root; i.e. not stable \\
\end{tabular}

Otherwise, $R^2$ and $\hat{\beta}_{j}$'s are misleading

\subsection{$AR(1)$ and Unit Root Processes}
$AR(1)$ model (Auto Regressive of order 1): 
\begin{align*}
y_t = \rho y_{t-1} + u_{t}
\end{align*}
Stable if $\vert\rho\vert<1$; Unit Root if $\vert\rho\vert\geq 1$

``Non-stationary,'' ``Unit Root,'' ``Integrated'' are all synonymous
\subsubsection{Correcting for Non-stationarity}
Easiest way is to take a first difference:

\medskip{}

\begin{tabular}{@{}p{\the\MyLen}%
                @{}p{\linewidth-\the\MyLen}@{}}
First difference: & Use  $\Delta y = y_{t}-y_{t-1}$ instead of $y_t$ \\
Test for unit root: & Augmented Dickey-Fuller (ADF) test \\
$H_0$ of ADF test: & $y$ has a unit root \\
\end{tabular}


\subsection{TS Forecasting}
A good forecast minimizes \textbf{forecasting error} $\hat{f}_t$:
\begin{align*}
\min_{f_t} E\left(e_{t+1}^{2}\vert I_t\right) = E\left[\left(y_{t+1}-f_t\right)^{2}\vert I_t\right]
\end{align*}
where $I_t$ is the \textbf{information set}

% \medskip{}

% Want a model that predicts the future as correctly as possible

\medskip{}

RMSE measures forecast performance (on future data): 
\begin{align*}
\text{Root Mean Squared Error} = \sqrt{\frac{1}{m}\sum_{h=0}^{m-1} \hat{e}^2_{T+h+1} }
\end{align*}

\medskip{}

Model with lowest RMSE is best forecast

\begin{itemize}
    \item Can choose $f_{t}$ in many ways
    \item Basic way: $\hat{y}_{T+1}$ from linear model
    \item ARIMA, ARMA-GARCH are cutting-edge models
\end{itemize}

\subsubsection{Granger causality}
$z$ \textbf{Granger causes} $y$ if, after controlling for past values of $y$, past values of $z$ help forecast $y_t$


\section{CLM violations}
\subsubsection{Heteroskedasticity}
\begin{itemize}
    \item Test: Breusch-Pagan or White tests ($H_0:$ homosk.)
    \item If $H_0$ rejected, SEs, t-, and F-stats are invalid
    \item Instead use heterosk.-robust SEs and t- and F-stats
\end{itemize}

\subsubsection{Serial correlation}
\begin{itemize}
    \item Test: Breusch-Godfrey test ($H_0:$ no serial corr.)
    \item If $H_0$ rejected, SEs, t-, and F-stats are invalid
    \item Instead use HAC SEs and t- and F-stats
    \item HAC: ``Heterosk. and Autocorrelation Consistent''
\end{itemize}

\subsubsection{Measurement error}
\begin{itemize}
    \item Measurement error in $x$ can be a violation of A4
    \item \textbf{Attenuation bias:} $\beta_j$ biased towards 0
\end{itemize}

\subsubsection{Omitted Variable Bias}
When an important $x$ is excluded: \textbf{omitted variable bias}.

\medskip{}

Bias depends on two forces:
\begin{enumerate}
    \item Partial effect of $x_2$ on $y$ (i.e. $\beta_2$)
    \item Correlation between $x_2$ and $x_1$
\end{enumerate}

Which direction does the bias go?
\begin{center}
\begin{threeparttable}
\begin{tabular}{@{}l@{}l@{}l@{}}
\toprule
& $Corr(x_{1},x_{2})>0$\phantom{\%\%\%} & $Corr(x_{1},x_{2})<0$\phantom{\%\%\%} \\ \midrule
$\beta _{2}>0$\phantom{\%\%\%} & Positive Bias & Negative Bias \\
$\beta _{2}<0$ & Negative Bias & Positive Bias \\ \bottomrule
\end{tabular}
\footnotesize Note: ``Positive bias'' means $\beta_1$ is too big; \par
``Negative bias'' means $\beta_1$ is too small
\end{threeparttable}
\end{center}

\subsection{How to resolve $E\left(u\vert\mathbf{x}\right)\neq 0$}
How can we get unbiased $\hat{\beta}_j$'s when $E\left(u\vert\mathbf{x}\right)\neq 0$?

\medskip{}

\begin{itemize}
    \item Include lagged $y$ as a regressor
    \item Include proxy variables for omitted ones
    \item Use instrumental variables
    \item Use a natural experiment (e.g. diff-in-diff)
    \item Use panel data
\end{itemize}



\subsubsection{Instrumental Variables (IV)}
A variable $z$, called the instrument, satisfies:

\medskip{}

\begin{enumerate}
    \item $cov\left(z,u\right) = 0$ (\textbf{not} testable)
    \item $cov\left(z,x\right) \neq 0$ (testable)
\end{enumerate}

\medskip{}

$z$ typically comes from a \textbf{natural experiment}

\begin{align*}
    \hat{\beta}_{IV} = \frac{cov\left(z,y\right)}{cov\left(z,x\right)}
\end{align*}

\begin{itemize}
    \item SE's much larger when using IV compared to OLS
    \item Be aware of \textbf{weak instruments}
\end{itemize}

\textbf{When there are multiple instruments:}

\begin{itemize}
    \item use Two-stage least squares (2SLS)
    \item exclude at least as many $z$'s as endogenous $x$'s
    \item[1st stage:] regress endogenous $x$ on $z$'s and exogenous $x$'s
    \item[2nd stage:] regress $y$ on $\hat{x}$ and exogenous $x$'s
\end{itemize}

\textbf{Test for weak instruments:} Instrument is weak if

\begin{itemize}
    \item 1st stage $F$ stat $< 10$
    \item or 1st stage $\vert t \vert < \sqrt{10}\approx 3.2$
\end{itemize}

\subsubsection{Difference in Differences (DiD)}
Can get causal effects from \textbf{pooled cross sectional data}

\medskip{}

A nat. experiment divides units into treatment, control groups

\begin{align*}
    y_{it} = \beta_0 + \delta_0 d2_t + \beta_1 dT_{it} + \delta_1 d2_{t}\times dT_{it} + u_{it}
\end{align*}
where
\begin{itemize}
    \item $d2_t=$ dummy for being in time period 2
    \item $dT_{it}=$ dummy for being in the treatment group
    \item $\hat{\delta}_1=$ \textbf{difference in differences}
\end{itemize}
\begin{align*}
    \hat{\delta}_1 = \left(\overline{y}_{treat,2}-\overline{y}_{control,2}\right) - \left(\overline{y}_{treat,1}-\overline{y}_{control,1}\right)
\end{align*}

\medskip{}


\textbf{Extensions:}
\begin{itemize}
    \item Can also include $x$'s in the model
    \item Can also use with more than 2 time periods
    \item $\hat{\delta}_1$ has same interpretation, different math formula
\end{itemize}

\medskip{}

\textbf{Validity:}
\begin{itemize}
    \item Need $y$ changing across time and treatment for reasons only due to the policy
    \item a.k.a. \textbf{parallel trends assumption}
\end{itemize}


\subsubsection{Panel data}
Follow same sample of units over multiple time periods
\begin{align*}
    y_{it} = \beta_0 + \beta_1 x_{it1} + \cdots + \beta_k x_{itk} + \underbrace{a_i + u_{it}}_{\nu_{it}}
\end{align*}
\begin{itemize}
    \item $\nu_{it}=$ \textbf{composite error}
    \item $a_i=$  unit-specific unobservables
    \item $u_{it}=$ idiosyncratic error
    \item Allow $E\left(a\vert\mathbf{x}\right)\neq 0$
    \item Maintain $E\left(u\vert\mathbf{x}\right)= 0$
\end{itemize}

\medskip{}

Four different methods of estimating $\beta_j$'s:
\begin{enumerate}
    \item Pooled OLS (i.e. ignore composite error)
    \item First differences (FD):
\begin{align*}
    \Delta y_{i} = \beta_1 \Delta x_{i1} + \cdots + \Delta \beta_k x_{ik} + \Delta u_{i}
\end{align*}
estimated via Pooled OLS on transformed data
    \item Fixed effects (FE):
\begin{align*}
    y_{it}-\overline{y}_{i} &= \beta_1 \left(x_{it1}-\overline{x}_{i1}\right) + \cdots \\
    &+\beta_k \left( x_{itk}-\overline{x}_{ik}\right) + \left(u_{it}-\overline{u}_{i}\right)
\end{align*}
estimated via Pooled OLS on transformed data
    \item Random effects (RE):
\begin{align*}
    y_{it}-\theta\overline{y}_{i} &= \beta_0 \left(1-\theta\right) + \beta_1 \left(x_{it1}-\theta\overline{x}_{i1}\right) + \cdots \\
    &+\beta_k \left( x_{itk}-\theta\overline{x}_{ik}\right) + \left(\nu_{it}-\theta\overline{\nu}_{i}\right)
\end{align*}
estimated via FGLS, where
\begin{align*}
    \theta &= 1-\sqrt{\frac{\sigma^2_u}{\sigma^2_u + T\sigma^2_a}} \\
    \hat{\beta}_{RE} \rightarrow \hat{\beta}_{FE} & \text{ as } \theta \rightarrow 1 \\
    \hat{\beta}_{RE} \rightarrow \hat{\beta}_{POLS} & \text{ as } \theta \rightarrow 0 \\
\end{align*}
\textbf{RE assumes $E\left(a\vert\mathbf{x}\right)= 0$}
\end{enumerate}

\paragraph{Cluster-robust SEs}
\begin{itemize}
    \item Serial correlation of $\nu_{it}$ is a problem
    \item Use \textbf{cluster-robust} SEs
    \item These correct for serial corr. and heterosk.
    \item Cluster at the unit level
\end{itemize}


\textcolor{white}{\lipsum[1-4]}

\rule{0.3\linewidth}{0.25pt}
\scriptsize

Layout by Winston Chang, \href{http://wch.github.io/latexsheet/}{http://wch.github.io/latexsheet/}

%---------------------------------------------------------------------------------------------------
% NEXT: Omitted Variable Bias; Hypothesis Testing; Dummy Variables; LPMs
%---------------------------------------------------------------------------------------------------

% \begin{multicols}{2}
% % \verb!\part{!\textit{title}\verb!}!  \\
% % \verb!\chapter{!\textit{title}\verb!}!  \\
% % \verb!\section{!\textit{title}\verb!}!  \\
% % \verb!\subsection{!\textit{title}\verb!}!  \\
% % \verb!\subsubsection{!\textit{title}\verb!}!  \\
% % \verb!\paragraph{!\textit{title}\verb!}!  \\
% % \verb!\subparagraph{!\textit{title}\verb!}!
% % \end{multicols}
% % {\raggedright
% % Use \verb!\setcounter{secnumdepth}{!$x$\verb!}! suppresses heading
% % numbers of depth $>x$, where \verb!chapter! has depth 0.
% % Use a \texttt{*}, as in \verb!\section*{!\textit{title}\verb!}!,
% % to not number a particular item---these items will also not appear
% % in the table of contents.
% % }

% \subsection{Text environments}
% \settowidth{\MyLen}{\texttt{.begin.quotation.}}
% \begin{tabular}{@{}p{\the\MyLen}%
                % @{}p{\linewidth-\the\MyLen}@{}}
% \verb!\begin{comment}!    &  Comment (not printed). Requires \texttt{verbatim} package. \\
% \verb!\begin{quote}!      &  Indented quotation block. \\
% \verb!\begin{quotation}!  &  Like \texttt{quote} with indented paragraphs. \\
% \verb!\begin{verse}!      &  Quotation block for verse.
% \end{tabular}

% \subsection{Lists}
% \settowidth{\MyLen}{\texttt{.begin.description.}}
% \begin{tabular}{@{}p{\the\MyLen}%
                % @{}p{\linewidth-\the\MyLen}@{}}
% \verb!\begin{enumerate}!        &  Numbered list. \\
% \verb!\begin{itemize}!          &  Bulleted list. \\
% \verb!\begin{description}!      &  Description list. \\
% \verb!\item! \textit{text}      &  Add an item. \\
% \verb!\item[!\textit{x}\verb!]! \textit{text}
                                % &  Use \textit{x} instead of normal
                        % bullet or number.  Required for descriptions. \\
% \end{tabular}




% \subsection{References}
% \settowidth{\MyLen}{\texttt{.pageref.marker..}}
% \begin{tabular}{@{}p{\the\MyLen}%
                % @{}p{\linewidth-\the\MyLen}@{}}
% \verb!\label{!\textit{marker}\verb!}!   &  Set a marker for cross-reference, 
                          % often of the form \verb!\label{sec:item}!. \\
% \verb!\ref{!\textit{marker}\verb!}!   &  Give section/body number of marker. \\
% \verb!\pageref{!\textit{marker}\verb!}! &  Give page number of marker. \\
% \verb!\footnote{!\textit{text}\verb!}!  &  Print footnote at bottom of page. \\
% \end{tabular}


% \subsection{Floating bodies}
% \settowidth{\MyLen}{\texttt{.begin.equation..place.}}
% \begin{tabular}{@{}p{\the\MyLen}%
                % @{}p{\linewidth-\the\MyLen}@{}}
% \verb!\begin{table}[!\textit{place}\verb!]!     &  Add numbered table. \\
% \verb!\begin{figure}[!\textit{place}\verb!]!    &  Add numbered figure. \\
% \verb!\begin{equation}[!\textit{place}\verb!]!  &  Add numbered equation. \\
% \verb!\caption{!\textit{text}\verb!}!           &  Caption for the body. \\
% \end{tabular}

% The \textit{place} is a list valid placements for the body.  \texttt{t}=top,
% \texttt{h}=here, \texttt{b}=bottom, \texttt{p}=separate page, \texttt{!}=place even if ugly.  Captions and label markers should be within the environment.

% %---------------------------------------------------------------------------

% \section{Text properties}

% \subsection{Font face}
% \newcommand{\FontCmd}[3]{\PBS\verb!\#1{!\textit{text}\verb!}!  \> %
                         % \verb!{\#2 !\textit{text}\verb!}! \> %
                         % \#1{#3}}
% \begin{tabular}{@{}l@{}l@{}l@{}}
% \textit{Command} & \textit{Declaration} & \textit{Effect} \\
% \verb!\textrm{!\textit{text}\verb!}!                    & %
        % \verb!{\rmfamily !\textit{text}\verb!}!               & %
        % \textrm{Roman family} \\
% \verb!\textsf{!\textit{text}\verb!}!                    & %
        % \verb!{\sffamily !\textit{text}\verb!}!               & %
        % \textsf{Sans serif family} \\
% \verb!\texttt{!\textit{text}\verb!}!                    & %
        % \verb!{\ttfamily !\textit{text}\verb!}!               & %
        % \texttt{Typewriter family} \\
% \verb!\textmd{!\textit{text}\verb!}!                    & %
        % \verb!{\mdseries !\textit{text}\verb!}!               & %
        % \textmd{Medium series} \\
% \verb!\textbf{!\textit{text}\verb!}!                    & %
        % \verb!{\bfseries !\textit{text}\verb!}!               & %
        % \textbf{Bold series} \\
% \verb!\textup{!\textit{text}\verb!}!                    & %
        % \verb!{\upshape !\textit{text}\verb!}!               & %
        % \textup{Upright shape} \\
% \verb!\textit{!\textit{text}\verb!}!                    & %
        % \verb!{\itshape !\textit{text}\verb!}!               & %
        % \textit{Italic shape} \\
% \verb!\textsl{!\textit{text}\verb!}!                    & %
        % \verb!{\slshape !\textit{text}\verb!}!               & %
        % \textsl{Slanted shape} \\
% \verb!\textsc{!\textit{text}\verb!}!                    & %
        % \verb!{\scshape !\textit{text}\verb!}!               & %
        % \textsc{Small Caps shape} \\
% \verb!\emph{!\textit{text}\verb!}!                      & %
        % \verb!{\em !\textit{text}\verb!}!               & %
        % \emph{Emphasized} \\
% \verb!\textnormal{!\textit{text}\verb!}!                & %
        % \verb!{\normalfont !\textit{text}\verb!}!       & %
        % \textnormal{Document font} \\
% \verb!\underline{!\textit{text}\verb!}!                 & %
                                                        % & %
        % \underline{Underline}
% \end{tabular}

% The command (t\textit{tt}t) form handles spacing better than the
% declaration (t{\itshape tt}t) form.

% \subsection{Font size}
% \setlength{\columnsep}{14pt} % Need to move columns apart a little
% \begin{multicols}{2}
% \begin{tabbing}
% \verb!\footnotesize!          \= \kill
% \verb!\tiny!                  \>  \tiny{tiny} \\
% \verb!\scriptsize!            \>  \scriptsize{scriptsize} \\
% \verb!\footnotesize!          \>  \footnotesize{footnotesize} \\
% \verb!\small!                 \>  \small{small} \\
% \verb!\normalsize!            \>  \normalsize{normalsize} \\
% \verb!\large!                 \>  \large{large} \\
% \verb!\Large!                 \=  \Large{Large} \\  % Tab hack for new column
% \verb!\LARGE!                 \>  \LARGE{LARGE} \\
% \verb!\huge!                  \>  \huge{huge} \\
% \verb!\Huge!                  \>  \Huge{Huge}
% \end{tabbing}
% \end{multicols}
% \setlength{\columnsep}{1pt} % Set column separation back

% These are declarations and should be used in the form
% \verb!{\small! \ldots\verb!}!, or without braces to affect the entire
% document.


% \subsection{Verbatim text}

% \settowidth{\MyLen}{\texttt{.begin.verbatim..} }
% \begin{tabular}{@{}p{\the\MyLen}%
                % @{}p{\linewidth-\the\MyLen}@{}}
% \verb@\begin{verbatim}@ & Verbatim environment. \\
% \verb@\begin{verbatim*}@ & Spaces are shown as \verb*@ @. \\
% \verb@\verb!text!@ & Text between the delimiting characters (in this case %
                      % `\texttt{!}') is verbatim.
% \end{tabular}


% \subsection{Justification}
% \begin{tabular}{@{}ll@{}}
% \textit{Environment}  &  \textit{Declaration}  \\
% \verb!\begin{center}!      & \verb!\centering!  \\
% \verb!\begin{flushleft}!  & \verb!\raggedright! \\
% \verb!\begin{flushright}! & \verb!\raggedleft!  \\
% \end{tabular}

% \subsection{Miscellaneous}
% \verb!\linespread{!$x$\verb!}! changes the line spacing by the
% multiplier $x$.





% \section{Text-mode symbols}

% \subsection{Symbols}
% \begin{tabular}{@{}l@{\hspace{1em}}l@{\hspace{2em}}l@{\hspace{1em}}l@{\hspace{2em}}l@{\hspace{1em}}l@{\hspace{2em}}l@{\hspace{1em}}l@{}}
% \&              &  \verb!\&! &
% \_              &  \verb!\_! &
% \ldots          &  \verb!\ldots! &
% \textbullet     &  \verb!\textbullet! \\
% \$              &  \verb!\$! &
% \^{}            &  \verb!\^{}! &
% \textbar        &  \verb!\textbar! &
% \textbackslash  &  \verb!\textbackslash! \\
% \%              &  \verb!\%! &
% \~{}            &  \verb!\~{}! &
% \#              &  \verb!\#! &
% \S              &  \verb!\S! \\
% \end{tabular}

% \subsection{Accents}
% \begin{tabular}{@{}l@{\ }l|l@{\ }l|l@{\ }l|l@{\ }l|l@{\ }l@{}}
% \`o   & \verb!\`o! &
% \'o   & \verb!\'o! &
% \^o   & \verb!\^o! &
% \~o   & \verb!\~o! &
% \=o   & \verb!\=o! \\
% \.o   & \verb!\.o! &
% \"o   & \verb!\"o! &
% \c o  & \verb!\c o! &
% \v o  & \verb!\v o! &
% \H o  & \verb!\H o! \\
% \c c  & \verb!\c c! &
% \d o  & \verb!\d o! &
% \b o  & \verb!\b o! &
% \t oo & \verb!\t oo! &
% \oe   & \verb!\oe! \\
% \OE   & \verb!\OE! &
% \ae   & \verb!\ae! &
% \AE   & \verb!\AE! &
% \aa   & \verb!\aa! &
% \AA   & \verb!\AA! \\
% \o    & \verb!\o! &
% \O    & \verb!\O! &
% \l    & \verb!\l! &
% \L    & \verb!\L! &
% \i    & \verb!\i! \\
% \j    & \verb!\j! &
% !`    & \verb!~`! &
% ?`    & \verb!?`! &
% \end{tabular}


% \subsection{Delimiters}
% \begin{tabular}{@{}l@{\ }ll@{\ }ll@{\ }ll@{\ }ll@{\ }ll@{\ }l@{}}
% `       & \verb!`!  &
% ``      & \verb!``! &
% \{      & \verb!\{! &
% \lbrack & \verb![! &
% (       & \verb!(! &
% \textless  &  \verb!\textless! \\
% '       & \verb!'!  &
% ''      & \verb!''! &
% \}      & \verb!\}! &
% \rbrack & \verb!]! &
% )       & \verb!)! &
% \textgreater  &  \verb!\textgreater! \\
% \end{tabular}

% \subsection{Dashes}
% \begin{tabular}{@{}llll@{}}
% \textit{Name} & \textit{Source} & \textit{Example} & \textit{Usage} \\
% hyphen  & \verb!-!   & X-ray          & In words. \\
% en-dash & \verb!--!  & 1--5           & Between numbers. \\
% em-dash & \verb!---! & Yes---or no?    & Punctuation.
% \end{tabular}


% \subsection{Line and page breaks}
% \settowidth{\MyLen}{\texttt{.pagebreak} }
% \begin{tabular}{@{}p{\the\MyLen}%
                % @{}p{\linewidth-\the\MyLen}@{}}
% \verb!\\!          &  Begin new line without new paragraph.  \\
% \verb!\\*!         &  Prohibit pagebreak after linebreak. \\
% \verb!\kill!       &  Don't print current line. \\
% \verb!\pagebreak!  &  Start new page. \\
% \verb!\noindent!   &  Do not indent current line.
% \end{tabular}


% \subsection{Miscellaneous}
% \settowidth{\MyLen}{\texttt{.rule.w..h.} }
% \begin{tabular}{@{}p{\the\MyLen}%
                % @{}p{\linewidth-\the\MyLen}@{}}
% \verb!\today!  &  \today. \\
% \verb!$\sim$!  &  Prints $\sim$ instead of \verb!\~{}!, which makes \~{}. \\
% \verb!~!       &  Space, disallow linebreak (\verb!W.J.~Clinton!). \\
% \verb!\@.!     &  Indicate that the \verb!.! ends a sentence when following
                        % an uppercase letter. \\
% \verb!\hspace{!$l$\verb!}! & Horizontal space of length $l$
                                % (Ex: $l=\mathtt{20pt}$). \\
% \verb!\vspace{!$l$\verb!}! & Vertical space of length $l$. \\
% \verb!\rule{!$w$\verb!}{!$h$\verb!}! & Line of width $w$ and height $h$. \\
% \end{tabular}



% \section{Tabular environments}

% \subsection{\texttt{tabbing} environment}
% \begin{tabular}{@{}l@{\hspace{1.5ex}}l@{\hspace{10ex}}l@{\hspace{1.5ex}}l@{}}
% \verb!\=!  &   Set tab stop. &
% \verb!\>!  &   Go to tab stop.
% \end{tabular}

% Tab stops can be set on ``invisible'' lines with \verb!\kill!
% at the end of the line.  Normally \verb!\\! is used to separate lines.


% \subsection{\texttt{tabular} environment}
% \verb!\begin{array}[!\textit{pos}\verb!]{!\textit{cols}\verb!}!   \\
% \verb!\begin{tabular}[!\textit{pos}\verb!]{!\textit{cols}\verb!}! \\
% \verb!\begin{tabular*}{!\textit{width}\verb!}[!\textit{pos}\verb!]{!\textit{cols}\verb!}!


% \subsubsection{\texttt{tabular} column specification}
% \settowidth{\MyLen}{\texttt{p}\{\textit{width}\} \ }
% \begin{tabular}{@{}p{\the\MyLen}@{}p{\linewidth-\the\MyLen}@{}}
% \texttt{l}    &   Left-justified column.  \\
% \texttt{c}    &   Centered column.  \\
% \texttt{r}    &   Right-justified column. \\
% \verb!p{!\textit{width}\verb!}!  &  Same as %
                              % \verb!\parbox[t]{!\textit{width}\verb!}!. \\ 
% \verb!@{!\textit{decl}\verb!}!   &  Insert \textit{decl} instead of
                                    % inter-column space. \\
% \verb!|!      &   Inserts a vertical line between columns. 
% \end{tabular}


% \subsubsection{\texttt{tabular} elements}
% \settowidth{\MyLen}{\texttt{.cline.x-y..}}
% \begin{tabular}{@{}p{\the\MyLen}@{}p{\linewidth-\the\MyLen}@{}}
% \verb!\hline!           &  Horizontal line between rows.  \\
% \verb!\cline{!$x$\verb!-!$y$\verb!}!  &
                        % Horizontal line across columns $x$ through $y$. \\
% \verb!\multicolumn{!\textit{n}\verb!}{!\textit{cols}\verb!}{!\textit{text}\verb!}! \\
        % &  A cell that spans \textit{n} columns, with \textit{cols} column specification.
% \end{tabular}

% \section{Math mode}
% For inline math, use \verb!\(...\)! or \verb!$...$!.
% For displayed math, use \verb!\[...\]! or \verb!\begin{equation}!.

% \begin{tabular}{@{}l@{\hspace{1em}}l@{\hspace{2em}}l@{\hspace{1em}}l@{}}
% Superscript$^{x}$       &
% \verb!^{x}!             &  
% Subscript$_{x}$         &
% \verb!_{x}!             \\  
% $\frac{x}{y}$           &
% \verb!\frac{x}{y}!      &  
% $\sum_{k=1}^n$          &
% \verb!\sum_{k=1}^n!     \\  
% $\sqrt[n]{x}$           &
% \verb!\sqrt[n]{x}!      &  
% $\prod_{k=1}^n$         &
% \verb!\prod_{k=1}^n!    \\ 
% \end{tabular}

% \subsection{Math-mode symbols}

% % The ordering of these symbols is slightly odd.  This is because I had to put all the
% % long pieces of text in the same column (the right) for it all to fit properly.
% % Otherwise, it wouldn't be possible to fit four columns of symbols here.

% \begin{tabular}{@{}l@{\hspace{1ex}}l@{\hspace{1em}}l@{\hspace{1ex}}l@{\hspace{1em}}l@{\hspace{1ex}} l@{\hspace{1em}}l@{\hspace{1ex}}l@{}}
% $\leq$          &  \verb!\leq!  &
% $\geq$          &  \verb!\geq!  &
% $\neq$          &  \verb!\neq!  &
% $\approx$       &  \verb!\approx!  \\
% $\times$        &  \verb!\times!  &
% $\div$          &  \verb!\div!  &
% $\pm$           & \verb!\pm!  &
% $\cdot$         &  \verb!\cdot!  \\
% $^{\circ}$      & \verb!^{\circ}! &
% $\circ$         &  \verb!\circ!  &
% $\prime$        & \verb!\prime!  &
% $\cdots$        &  \verb!\cdots!  \\
% $\infty$        & \verb!\infty!  &
% $\neg$          & \verb!\neg!  &
% $\wedge$        & \verb!\wedge!  &
% $\vee$          & \verb!\vee!  \\
% $\supset$       & \verb!\supset!  &
% $\forall$       & \verb!\forall!  &
% $\in$           & \verb!\in!  &
% $\rightarrow$   &  \verb!\rightarrow! \\
% $\subset$       & \verb!\subset!  &
% $\exists$       & \verb!\exists!  &
% $\notin$        & \verb!\notin!  &
% $\Rightarrow$   &  \verb!\Rightarrow! \\
% $\cup$          & \verb!\cup!  &
% $\cap$          & \verb!\cap!  &
% $\mid$          & \verb!\mid!  &
% $\Leftrightarrow$   &  \verb!\Leftrightarrow! \\
% $\dot a$        & \verb!\dot a!  &
% $\hat a$        & \verb!\hat a!  &
% $\bar a$        & \verb!\bar a!  &
% $\tilde a$      & \verb!\tilde a!  \\

% $\alpha$        &  \verb!\alpha!  &
% $\beta$         &  \verb!\beta!  &
% $\gamma$        &  \verb!\gamma!  &
% $\delta$        &  \verb!\delta!  \\
% $\epsilon$      &  \verb!\epsilon!  &
% $\zeta$         &  \verb!\zeta!  &
% $\eta$          &  \verb!\eta!  &
% $\varepsilon$   &  \verb!\varepsilon!  \\
% $\theta$        &  \verb!\theta!  &
% $\iota$         &  \verb!\iota!  &
% $\kappa$        &  \verb!\kappa!  &
% $\vartheta$     &  \verb!\vartheta!  \\
% $\lambda$       &  \verb!\lambda!  &
% $\mu$           &  \verb!\mu!  &
% $\nu$           &  \verb!\nu!  &
% $\xi$           &  \verb!\xi!  \\
% $\pi$           &  \verb!\pi!  &
% $\rho$          &  \verb!\rho!  &
% $\sigma$        &  \verb!\sigma!  &
% $\tau$          &  \verb!\tau!  \\
% $\upsilon$      &  \verb!\upsilon!  &
% $\phi$          &  \verb!\phi!  &
% $\chi$          &  \verb!\chi!  &
% $\psi$          &  \verb!\psi!  \\
% $\omega$        &  \verb!\omega!  &
% $\Gamma$        &  \verb!\Gamma!  &
% $\Delta$        &  \verb!\Delta!  &
% $\Theta$        &  \verb!\Theta!  \\
% $\Lambda$       &  \verb!\Lambda!  &
% $\Xi$           &  \verb!\Xi!  &
% $\Pi$           &  \verb!\Pi!  &
% $\Sigma$        &  \verb!\Sigma!  \\
% $\Upsilon$      &  \verb!\Upsilon!  &
% $\Phi$          &  \verb!\Phi!  &
% $\Psi$          &  \verb!\Psi!  &
% $\Omega$        &  \verb!\Omega!  
% \end{tabular}
% \footnotesize

% %\subsection{Special symbols}
% %\begin{tabular}{@{}ll@{}}
% %$^{\circ}$  &  \verb!^{\circ}! Ex: $22^{\circ}\mathrm{C}$: \verb!$22^{\circ}\mathrm{C}$!.
% %\end{tabular}

% \section{Bibliography and citations}
% When using \BibTeX, you need to run \texttt{latex}, \texttt{bibtex},
% and \texttt{latex} twice more to resolve dependencies.

% \subsection{Citation types}
% \settowidth{\MyLen}{\texttt{.shortciteN.key..}}
% \begin{tabular}{@{}p{\the\MyLen}@{}p{\linewidth-\the\MyLen}@{}}
% \verb!\cite{!\textit{key}\verb!}!       &
        % Full author list and year. (Watson and Crick 1953) \\
% \verb!\citeA{!\textit{key}\verb!}!      &
        % Full author list. (Watson and Crick) \\
% \verb!\citeN{!\textit{key}\verb!}!      &
        % Full author list and year. Watson and Crick (1953) \\
% \verb!\shortcite{!\textit{key}\verb!}!  &
        % Abbreviated author list and year. ? \\
% \verb!\shortciteA{!\textit{key}\verb!}! &
        % Abbreviated author list. ? \\
% \verb!\shortciteN{!\textit{key}\verb!}! &
        % Abbreviated author list and year. ? \\
% \verb!\citeyear{!\textit{key}\verb!}!   &
        % Cite year only. (1953) \\
% \end{tabular}

% All the above have an \texttt{NP} variant without parentheses;
% Ex. \verb!\citeNP!.


% \subsection{\BibTeX\ entry types}
% \settowidth{\MyLen}{\texttt{.mastersthesis.}}
% \begin{tabular}{@{}p{\the\MyLen}@{}p{\linewidth-\the\MyLen}@{}}
% \verb!@article!         &  Journal or magazine article. \\
% \verb!@book!            &  Book with publisher. \\
% \verb!@booklet!         &  Book without publisher. \\
% \verb!@conference!      &  Article in conference proceedings. \\
% \verb!@inbook!          &  A part of a book and/or range of pages. \\
% \verb!@incollection!    &  A part of book with its own title. \\
% %\verb!@manual!          &  Technical documentation. \\
% %\verb!@mastersthesis!   &  Master's thesis. \\
% \verb!@misc!            &  If nothing else fits. \\
% \verb!@phdthesis!       &  PhD. thesis. \\
% \verb!@proceedings!     &  Proceedings of a conference. \\
% \verb!@techreport!      &  Tech report, usually numbered in series. \\
% \verb!@unpublished!     &  Unpublished. \\
% \end{tabular}

% \subsection{\BibTeX\ fields}
% \settowidth{\MyLen}{\texttt{organization.}}
% \begin{tabular}{@{}p{\the\MyLen}@{}p{\linewidth-\the\MyLen}@{}}
% \verb!address!         &  Address of publisher.  Not necessary for major
                                % publishers.  \\
% \verb!author!           &  Names of authors, of format .... \\
% \verb!booktitle!        &  Title of book when part of it is cited. \\
% \verb!chapter!          &  Chapter or section number. \\
% \verb!edition!          &  Edition of a book. \\
% \verb!editor!           &  Names of editors. \\
% \verb!institution!      &  Sponsoring institution of tech.\ report. \\
% \verb!journal!          &  Journal name. \\
% \verb!key!              &  Used for cross ref.\ when no author. \\
% \verb!month!            &  Month published. Use 3-letter abbreviation. \\
% \verb!note!             &  Any additional information. \\
% \verb!number!           &  Number of journal or magazine. \\
% \verb!organization!     &  Organization that sponsors a conference. \\
% \verb!pages!            &  Page range (\verb!2,6,9--12!). \\
% \verb!publisher!        &  Publisher's name. \\
% \verb!school!           &  Name of school (for thesis). \\
% \verb!series!           &  Name of series of books. \\
% \verb!title!            &  Title of work. \\
% \verb!type!             &  Type of tech.\ report, ex. ``Research Note''. \\
% \verb!volume!           &  Volume of a journal or book. \\
% \verb!year!             &  Year of publication. \\
% \end{tabular}
% Not all fields need to be filled.  See example below.

% \subsection{Common \BibTeX\ style files}
% \begin{tabular}{@{}l@{\hspace{1em}}l@{\hspace{3em}}l@{\hspace{1em}}l@{}}
% \verb!abbrv!    &  Standard &
% \verb!abstract! &  \texttt{alpha} with abstract \\
% \verb!alpha!    &  Standard &
% \verb!apa!      &  APA \\
% \verb!plain!    &  Standard &
% \verb!unsrt!    &  Unsorted \\
% \end{tabular}

% The \LaTeX\ document should have the following two lines just before
% \verb!\end{document}!, where \verb!bibfile.bib! is the name of the
% \BibTeX\ file.
% \begin{verbatim}
% \bibliographystyle{plain}
% \bibliography{bibfile}
% \end{verbatim}

% \subsection{\BibTeX\ example}
% The \BibTeX\ database goes in a file called
% \textit{file}\texttt{.bib}, which is processed with \verb!bibtex file!. 
% \begin{verbatim}
% @String{N = {Na\-ture}}
% @Article{WC:1953,
  % author  = {James Watson and Francis Crick},
  % title   = {A structure for Deoxyribose Nucleic Acid},
  % journal = N,
  % volume  = {171},
  % pages   = {737},
  % year    = 1953
% }
% \end{verbatim}


% \section{Sample \LaTeX\ document}
% \begin{verbatim}
% \documentclass[11pt]{article}
% \usepackage{fullpage}
% \title{Template}
% \author{Name}
% \begin{document}
% \maketitle

% \section{section}
% \subsection*{subsection without number}
% text \textbf{bold text} text. Some math: $2+2=5$
% \subsection{subsection}
% text \emph{emphasized text} text. \cite{WC:1953}
% discovered the structure of DNA.

% A table:
% \begin{table}[!th]
% \begin{tabular}{|l|c|r|}
% \hline
% first  &  row  &  data \\
% second &  row  &  data \\
% \hline
% \end{tabular}
% \caption{This is the caption}
% \label{ex:table}
% \end{table}

% The table is numbered \ref{ex:table}.
% \end{document}
% \end{verbatim}

\end{multicols}
\end{document}
